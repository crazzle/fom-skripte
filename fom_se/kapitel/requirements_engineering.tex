%\part{Entwurfsprinzipien}

%----------------------------------------------------------------------------
\section{Requirements Engineering}
\begin{frame}[fragile]
	\frametitle{Requirements Engineering}
\huge Requirements Engineering
\end{frame}

\begin{frame}
\frametitle{Requirements Engineering}
	\begin{itemize}
		\item Anforderungen des Kunden an die Software sind der zentrale Faktor in einem Softwareprojekt!
		\item Wenn die Anforderungen des Kunden nicht erfüllt werden ist die Software wertlos
		\item Die systematische Ermittlung, Analyse und Spezifikation der Anforderungen wird als Systemanalyse
					oder Requirements Engineering bezeichnet
	\end{itemize}
	\begin{block}{Anforderung}
		Anforderungen (requirements) legen fest, was von einem Softwaresystem als Eigenschaft erwartet wird.
  \end{block}
\end{frame}

\begin{frame}
\frametitle{6 Dimensionen der Komplexität}
	Komplexität der Funktionen
	\begin{itemize}
		\item Software wird komplexer je mehr Funktionen sie besitzt
		\item Beispiel: Integrierte Büroanwendungen
		\begin{itemize}
			\item Tabellenkalkulation, Präsentationen, Textverarbeitung, Datenbank
			\item 1000 - 1500 Funktionen
		\end{itemize}
	\end{itemize}
	\bigskip
	Komplexität der Daten
	\begin{itemize}
		\item Wird durch Vielzahl von Datenstrukturen oder sehr komplexen Datenstrukturen definiert
		\item Beispiel: DB-orientierte Systeme, Datawarehouses, Big Data Applikationen
	\end{itemize}
\end{frame}

\begin{frame}
\frametitle{6 Dimensionen der Komplexität}
	Komplexität der Algorithmen
	\begin{itemize}
		\item Systeme die komplexe numerische Berechnungen durchführen
	\end{itemize}
	\bigskip
	Komplexität des zeitabhängigen Verhaltens
	\begin{itemize}
		\item Wird durch viele nebenläufige Prozesse, gegenseitiges Ausschließen oder Zeitbedingungen gekennzeichnet
		\item Beispiele sind Betriebssysteme, Prozesssteuerungen, verteilte Systeme
	\end{itemize}
\end{frame}

\begin{frame}
\frametitle{6 Dimensionen der Komplexität}
	Komplexität der Systemumgebung
	\begin{itemize}
		\item Systeme bei denen es im wesentlichen auf das Zusammenwirken mit dem Gesamtsystem ankommt
		\item Beispiel: Embedded Systems (Flugzeugsteuerungen, Fahrzeugsysteme), IOT Lösungen
	\end{itemize}
	\bigskip
	Komplexität der Benutzeroberfläche
	\begin{itemize}
		\item Systeme mit komplexer Interaktion zwischen Anwender und Computersystem
		\item Beispiel: CAD-Systeme, CASE-Syteme, Büro- und Enterpriseanwendungen
	\end{itemize}
\end{frame}

\begin{frame}
\frametitle{6 Dimensionen der Komplexität}
	Das Modellierungskonzept ist entsprechend der Komplexität des zu entwickelnden Produkts auszuwählen.
	Weist das zugrundeliegende Produkt beispielsweise eine hohe Komplexität der Funktionen auf, so muss das verwendete
	Konzept es erlauben eine Vielzahl von Funktionen zu modellieren.
\end{frame}

\begin{frame}
\frametitle{Analyse und Entwurf mit UML}
	\begin{columns}
			\begin{column}{.5\textwidth}
			\small
			\begin{itemize}
				\item Unified Modeling Language
				\item Sprache zur objektorientierten Modellierung
				\item Standardisiert von der OMG
				\item Auch zur Implementierung genutzt
			\end{itemize}
			\normalsize
			\end{column}
			\begin{column}{.5\textwidth}
				\center
				\includegraphics[width=1\textwidth,
				keepaspectratio=true]{bilder/uml_example.png}
			\end{column}
	\end{columns}
\end{frame}

\begin{frame}
\frametitle{Analyse und Entwurf mit UML}
		Berücksichtigung aktueller Trends der Softwareentwicklung
		\begin{itemize}
			\item Patterns
			\item Komponentenorientiertes Design
			\item Anwendungsfälle modellieren
			\item Ausführbarkeit von Modellen
		\end{itemize}
\end{frame}

\begin{frame}
\frametitle{Analyse und Entwurf mit UML}
	OOA (Objektorientierte Analyse) = Anwendersicht
	\begin{itemize}
		\item Modell der Anforderungen
	\end{itemize}
	\bigskip
	OOD (Objektorientiertes Design) = Entwicklersicht
	\begin{itemize}
	 \item Details und Randbedingungen
	\end{itemize}
  \bigskip
	OOP (Objektorientierte Progr.) = Entwicklersicht
	\begin{itemize}
		\item Implementierung
	\end{itemize}
\end{frame}

\begin{frame}
\frametitle{Exkurs OOP: Objekte, Methoden und Attribute}
	\center
	\includegraphics[width=0.9\textwidth, keepaspectratio=true]{bilder/kuh.png}
\end{frame}

\begin{frame}
\frametitle{Exkurs OOP: Klassen als Objektbauplan}
	\begin{columns}
	    \begin{column}{.6\textwidth}
			\small
			\begin{itemize}
			  \item Klasse fungiert als Bauplan f\"ur Objekte
			  \item Aus diesem Bauplan lassen sich beliebig viele Objekte erzeugen
			  \item Objekte aus einer Klasse besitzen die gleichen Methoden
			  \item Objekte aus einer Klasse besitzen die gleichen Attribute \\(Werte
			  k\"onnen jedoch variieren)
			  \item Klassen k\"onnen durch Vererbung weiter spezialisiert werden
			\end{itemize}
			\normalsize
	    \end{column}
	    \begin{column}{.4\textwidth}
	   		\center
			\includegraphics[width=\textwidth,
			keepaspectratio=true]{bilder/kuh_klasse.png}
	    \end{column}
	\end{columns}
\end{frame}

\begin{frame}
\frametitle{Exkurs OOP: Beziehungen zwischen Objekten}
\begin{columns}
	    \begin{column}{.6\textwidth}
			\small
			\begin{itemize}
			  \item Zwischen Objekten k\"onnen Beziehungen existieren
			  \begin{item}
			  		Beziehung zwischen Bauer und Kuh:
					\begin{enumerate}
					  \item \tiny Der Bauer kann 0 bis n K\"uhe besitzen
					  \item \tiny Eine Kuh geh\"ort genau einem Bauer
					  \item \tiny Keine Kuh kann einen Bauer besitzen \\ (gerichtete Beziehung)
					\end{enumerate}
			  \end{item}
			\end{itemize}
			\normalsize
	    \end{column}
	    \begin{column}{.4\textwidth}
	   		\center
			\includegraphics[width=0.9\textwidth,
			keepaspectratio=true]{bilder/asso_kuh.png}
	    \end{column}
	\end{columns}
\end{frame}

\begin{frame}
\frametitle{Exkurs OOP: Methodenaufrufe}
\begin{columns}
	    \begin{column}{.6\textwidth}
			\small
			\begin{itemize}
			  \item Objekte k\"onnen sich untereinander Botschaften senden
			  \item Damit wird der Empf\"anger aufgefordert etwas auszuf\"uhren
			  \begin{item}
			  		Erweitern wir die Klasse ''Kuh''
					\begin{enumerate}
					  \item \tiny Kuh bekommt die Methode: ''gebeMilch''
					  \item \tiny ''gebeMilch'' gibt ein Objekt vom Typ ''Milchtyp'' zur\"uck
					\end{enumerate}
			  \end{item}
			  \item Der Bauer l\"ost durch senden einer Botschaft die Methode
			  ''gebeMilch'' aus
			  \item Die Kuh f\"uhrt die Methode aus und gibt Milchtyp-Objekt zur\"uck
			\end{itemize}
			\normalsize
	    \end{column}
	    \begin{column}{.4\textwidth}
	   		\center
			\includegraphics[width=1\textwidth,
			keepaspectratio=true]{bilder/methodenaufruf.png}
	    \end{column}
	\end{columns}
\end{frame}

\subsection{Use Cases}
\begin{frame}
\frametitle{Use Cases}
	\begin{block}{Use Case}
		Ein Use Case beschreibt Szenarien einzelner Interaktionen zwischen dem Nutzer und dem System.
	\end{block}
	\bigskip
	Zielsetzung
	\begin{itemize}
		\item Spezifikation der Funktionalität des Systems
		\item Identifikation: Wer / was arbeitet mit dem System
		\item Trennung zwischen dem was das System leistet und was außerhalb bearbeitet wird
		\item Zergliederung in Einheiten mit Akteuren und den zugehörigen Anwendungsfällen
	\end{itemize}
\end{frame}

\begin{frame}
\frametitle{Use Case Beispiel}
	\center
	\includegraphics[width=1\textwidth,
	keepaspectratio=true]{bilder/use_case_beispiel.png}
\end{frame}

\begin{frame}
\frametitle{Use Cases}
	Beschreiben die Interaktionen zwischen den Akteuren und dem System
	\begin{itemize}
		\item Wie wird das System verwendet?
		\item Welches Verhalten zeigt das System nach außen?
		\item Use Cases werden stets von einem Akteur angestoßen,
					um bestimmte Funktionalität des Systems zu nutzen
	\end{itemize}
\end{frame}

\begin{frame}
\frametitle{Use Cases}
	Beschreiben alle Abläufe die zur Durchführung nötig sind
	\begin{itemize}
		\item Nur Abläufe und Schritte die nach außen sichtbar sind
		\item Keine internen Strukturen
	\end{itemize}
\end{frame}

\begin{frame}
\frametitle{Use Cases}
	Sind in sich abgeschlossen
	\begin{itemize}
		\item Alle Abläufe sind vollständig enthalten
		\item Jeder Use Case kann für sich allein bestehen und ist für sich allein sinnvoll
		\item Jeder Use Case liefert ein Resultat für mindestens einen Akteur
	\end{itemize}
	\bigskip
	Alle Use Cases modellieren in ihrer Gesamtheit das funktionale Verhalten des Systems
\end{frame}

\begin{frame}
\frametitle{Elemente eines Use Case - Akteur}
	\begin{columns}
		\begin{column}{.6\textwidth}
				\begin{itemize}
					\item Ein Akteur ist jemand/etwas, der/das
					\begin{itemize}
						\item \ldots mit dem zu entwickelnden System zusammenarbeitet
						\item \ldots eine interaktion ausführt
					\end{itemize}
					\item Ist stets außerhalb des Systems
					\item Sind Abstraktionen (Personen/Rollen, Maschinen, Systeme)
					\item Legen die Systemgrenzen fest
				\end{itemize}
		\end{column}
		\begin{column}{.4\textwidth}
			\center
			\includegraphics[width=0.4\textwidth,
			keepaspectratio=true]{bilder/akteur.png}
		\end{column}
	\end{columns}
\end{frame}

\begin{frame}
\frametitle{Elemente eines Use Case - Use Case}
	\begin{columns}
		\begin{column}{.6\textwidth}
			\begin{itemize}
				\item Aktivitäten die mit dem System durchgeführt werden
				\item Aktivitäten die nach außen hin sichtbar sind
			\end{itemize}
		\end{column}
		\begin{column}{.4\textwidth}
			\center
			\includegraphics[width=0.6\textwidth,
			keepaspectratio=true]{bilder/use_case.png}
		\end{column}
	\end{columns}
\end{frame}

\begin{frame}
\frametitle{Elemente eines Use Case - Interaktionsbeziehungen}
	\begin{columns}
		\begin{column}{.4\textwidth}
			\begin{itemize}
				\item Zusammenhänge von Akteuren und ihren Use Cases
			\end{itemize}
		\end{column}
		\begin{column}{.6\textwidth}
			\center
			\includegraphics[width=0.7\textwidth,
			keepaspectratio=true]{bilder/interaktionsbeziehung.png}
		\end{column}
	\end{columns}
\end{frame}

\begin{frame}
\frametitle{Fortgeschrittene Konzepte - Include}
	\begin{columns}
		\begin{column}{.4\textwidth}
			\scriptsize
			\begin{itemize}
				\item Mehrere Use Cases besitzen gleiche Teile
				\item Gemeinsamer Teil wird in eigenen Use Case ausgelagert
				\item Wird bei ausführung an entsprechender Stelle aufgerufen
				\item Ausgelagerter Teil muss sinnvolle Einheit ergeben
				\item Gerichtet vom Basisfall zum gemeinsamen Teil
			\end{itemize}
			\normalsize
		\end{column}
		\begin{column}{.6\textwidth}
			\center
			\includegraphics[width=1\textwidth,
			keepaspectratio=true]{bilder/include.png}
		\end{column}
	\end{columns}
\end{frame}

\begin{frame}
\frametitle{Fortgeschrittene Konzepte - Extend}
	\begin{columns}
		\begin{column}{.4\textwidth}
			\scriptsize
			\begin{itemize}
				\item Use Case kann an definierten Punkten erweitert werden
				\item Erweiterung trennt definiertes Verhalten von Optionalem / Ergänzendem
				\item Basis Use Case ist auch ohne Erweiterung sinnvoll
				\item Gerichtet von Optionalem zu Basis Use Case
			\end{itemize}
			\normalsize
		\end{column}
		\begin{column}{.6\textwidth}
			\center
			\includegraphics[width=1\textwidth,
			keepaspectratio=true]{bilder/extend.png}
		\end{column}
	\end{columns}
\end{frame}

\begin{frame}
\frametitle{Fortgeschrittene Konzepte - Generalisierung von Use Cases}
	\begin{columns}
		\begin{column}{.35\textwidth}
			\scriptsize
			\begin{itemize}
				\item Möglich bei semantisch ähnlichen Funktionalitäten
				\item Unter-Use Case erbt alle Eigenschaften des Ober-Use Case
				\item Unter-Use Case kann modifiziert oder ergänzt werden
				\item Unter-Use Case kann an jeder Stelle anstatt des Ober-Use Case verwendet werden
			\end{itemize}
			\normalsize
		\end{column}
		\begin{column}{.65\textwidth}
			\center
			\includegraphics[width=1\textwidth,
			keepaspectratio=true]{bilder/generalisierung_use_case.png}
		\end{column}
	\end{columns}
\end{frame}

\begin{frame}
\frametitle{Fortgeschrittene Konzepte - Generalisierung Akteure}
	\begin{columns}
		\begin{column}{.35\textwidth}
			\scriptsize
			\begin{itemize}
				\item Unter-Akteur erbt alle Use Cases des Ober-Akteurs
				\item Unter-Akteur kann eigene Use Cases erhalten
			\end{itemize}
			\normalsize
		\end{column}
		\begin{column}{.65\textwidth}
			\center
			\includegraphics[width=1\textwidth,
			keepaspectratio=true]{bilder/generalisierung_akteur.png}
		\end{column}
	\end{columns}
\end{frame}

\begin{frame}
\frametitle{Fortgeschrittene Konzepte - Vorsicht}
	\begin{alertblock}{Verständlichkeit}
		Use Cases dienen der Kommunikation mit den Anwendern. Daher müssen sie von diesen
		verstanden werden können. Es ist nicht Ziel alle der zur Verfügung stehenden Konzepte
		zu verwenden, denn damit entsteht eine höhere Komplexität und somit erschwerte Lesbarkeit
		/ Verständlichkeit.
	\end{alertblock}
\end{frame}

\begin{frame}
\frametitle{Template zur Beschreibung}
	\scriptsize
	\begin{table}[]
		\begin{tabular}{l|l}
		 Use Case 				  & Name des Use Case\\
		 Zusammenfassung 		& Kurzbeschreibung\\
		 Akteure 				    & Akteure die den Use Case auslösen oder involviert sind\\
		 Vorbedingungen 		& Vorraussetzungen die zur Ausführung erfüllt sein müssen\\
		 Ablaufbeschreibung & Strukturierte Ablaufbeschreibung (alle Standardfälle)\\
		 Verwendung 				& Auflistung aller Include-Beziehungen\\
		 Erweiterung 				& Auflistung aller Extend-Beziehungen\\
		 Alternativen 			& Optionen und Alternative Ausführungen\\
		 Nachbedingungen 	  & Resultate die der Use Case liefert\\
		 Fehlschlag 				& Erwarteter Zustand wenn das Ziel nicht erreicht wird\\
		 Sonstiges 				  & Nutzen des Use Case; Probleme die der Use Case löst\\
		\end{tabular}
	\end{table}
\end{frame}

\begin{frame}
\frametitle{Übung 4.1}
	Untersuchen Sie dieses Use Case Diagramm auf Fehler. Berichtigen Sie das Diagramm derart,
	dass es einen korrekten Sachverhalt wiedergibt.
	\bigskip
	\center
	\includegraphics[width=0.7\textwidth,
	keepaspectratio=true]{bilder/use_case_aufg1.png}
\end{frame}

\ifloesung
\begin{frame}
\frametitle{Übung 4.1 - Lösung}
	\includegraphics[width=1\textwidth,
	keepaspectratio=true]{bilder/use_case_aufg1_lsg.png}
\end{frame}
\fi

\begin{frame}
\frametitle{Übung 4.2}
	Erstellen Sie ein Use Case Diagramm für das Anrufen (das herkömmliche Telefongespräch zwischen 2 Personen).
	Berücksichtigen Sie dabei die Möglichkeit einer Konferenzschaltung (Hinzufügen eines weiteren Gesprächspartners
	zum bestehenden Telefongespräch).

	Beschreiben Sie die einzelnen Use Cases aus Ihrem Diagramm. Dazu verwenden Sie das in der Vorlesung vorgestellte
	Template.
\end{frame}

\ifloesung
\begin{frame}
\frametitle{Übung 4.2 - Lösung}
	\includegraphics[width=1\textwidth,
	keepaspectratio=true]{bilder/use_case_aufg2_lsg.png}
	(Das Use Case Diagramm lässt sich beliebig erweitern)
\end{frame}
\fi

\begin{frame}
\frametitle{Übung 4.3}
	Beschreiben Sie die Benutzung eines Bankautomaten mit dem Ziel Geld abzuheben in Form
	eines Use Cases. Beachten Sie dabei, dass Use Cases nur Abläufe / Schritte enthalten,
	die nach außen sichtbar sind; sie enthalten keine internen Strukturen.

	Beschreiben Sie die einzelnen Use Cases aus Ihrem Diagramm. Dazu verwenden Sie das in der Vorlesung vorgestellte
	Template.
\end{frame}

\ifloesung
\begin{frame}
\frametitle{Übung 4.3 - Lösung}
	\includegraphics[width=1\textwidth,
	keepaspectratio=true]{bilder/use_case_aufg3_lsg.png}
\end{frame}
\fi

\begin{frame}
\frametitle{Übung 4.4}
	Beschreiben Sie die Verwaltung einer Bibliothek bei der Buchausleihe in Form
	eines Use Cases.

	Beschreiben Sie die einzelnen Use Cases aus Ihrem Diagramm. Dazu verwenden Sie das in der Vorlesung vorgestellte
	Template.
\end{frame}

\ifloesung
\begin{frame}
\frametitle{Übung 4.4 - Lösung}
	\includegraphics[width=0.7\textwidth,
	keepaspectratio=true]{bilder/use_case_aufg4_lsg.png}
\end{frame}
\fi

\begin{frame}
\frametitle{Use Cases}
	Vorteile
	\begin{itemize}
		\item Leichter Zugang zum System
		\item Besseres Verständnis des Systems (Abgrenzug innerhalb und außerhalb des Systems)
		\item Basis für Klassendefinition
		\item Erleichtert Kommunikation zwischen Entwickler und Anwender
		\item Basis für System- und Abnahmetests
		\item Grundlage für Benutzerdokumentation
	\end{itemize}
	\bigskip
	Nachteile
	\begin{itemize}
		\item Gefahr der funktionalen Zerlegung (zu viel kleine Use Cases)
		\item Kontrollstrukturen lassen sich nicht einfach darstellen (nur in Beschreibung)
		\item User Interface / User Experience lässt sich nicht abbilden
	\end{itemize}
\end{frame}

\subsection{Klassendiagramme}
\begin{frame}
\frametitle{Notation Klassen}
\begin{columns}
	\begin{column}{.35\textwidth}
		\scriptsize
		\begin{itemize}
			\item Als Rechteck dargestellt
			\item In beliebig viele Compartments unterteilt
			\item Inhalt der Compartments frei wählbar
			\item Verschiedene Sichten durch Verzicht auf Compartments möglich
		\end{itemize}
		\normalsize
	\end{column}
	\begin{column}{.65\textwidth}
		Konvention:
		\center
		\includegraphics[width=0.7\textwidth,
		keepaspectratio=true]{bilder/klasse_notation.png}
	\end{column}
\end{columns}
\end{frame}

\begin{frame}
\frametitle{Attribute}
	\begin{block}{Syntax}
		Sichtbarkeit / Name : Typ Multiplizität = Anfangswert \{Einschränkungen\}
		\\
		(Mit Ausnahme des Namens sind alle Angaben optional)
	\end{block}
	\begin{itemize}
		\item Beschreiben Eigenschaften einer Klasse
		\item Zeigen Daten die von Objekten angenommen werden können
		\item Jedes Attribut hat einen Namen (Substantiv, bsp. Farbe, Preis)
	  \item Notation Klassenattribut: \uline{Klassenattribut}
		\item Abgeleitetes Attribut wird aus anderen Attributswerten berechnet, Notation: /Attribut
	\end{itemize}
\end{frame}

\begin{frame}
\frametitle{Sichtbarkeit Attribute}
	\begin{itemize}
		\item public: +
		\item private: -
		\item package: $\sim$
		\item protected: \#
	\end{itemize}
\end{frame}

\begin{frame}
\frametitle{Multiplizität Attribute}
	Die Multiplizität legt die Unter- und Obergrenze der Anzahl der unter einem Attributsnamen
	abglegbaren Instanzen fest.\\
	\begin{itemize}
		\item 0..1 optionales Attribut (höchstens ein Wert)
		\item 1..1 zwingend, genau ein Wert
		\item 0..* optional beliebig
		\item 1..* mindestens einer, belieb viele
		\item n..m mindestens n, höchstens m
	\end{itemize}
\end{frame}

\begin{frame}
\frametitle{Eigenschaftswerte Attribute}
	\begin{itemize}
		\item readOnly: Attribute deren Wert nicht verändert werden darf (Konstanten)
		\item ordered: Legt Reihenfolge der Inhalte eines Attributs fest
		\item unique: Legt fest ob Attribut duplikatsfrei ist
		\item In der UML Spezifikation sind weitere Eigenschaftswerte enthalten
	\end{itemize}
\end{frame}

\begin{frame}
\frametitle{Beispiel Attribute}
	Beispiel
	\center
	\includegraphics[width=0.5\textwidth,
	keepaspectratio=true]{bilder/attribute_beispiel.png}
\end{frame}

\begin{frame}
\frametitle{Übung 4.5}
	Die Spezifikation ist fehlerhaft, identifizieren Sie die Fehler.
	\center
	\includegraphics[width=1\textwidth,
	keepaspectratio=true]{bilder/klassend_aufg5.png}
\end{frame}

\ifloesung
\begin{frame}
\frametitle{Übung 4.5 - Lösung}
	\center
	\includegraphics[width=1\textwidth,
	keepaspectratio=true]{bilder/klassend_aufg5_lsg.png}
\end{frame}
\fi

\begin{frame}
\frametitle{Operationen}
	\begin{block}{Syntax}
		Sichtbarkeit Name ( Parameter ) : Rückgabetyp \{Einschränkungen\}
	\end{block}
	\begin{itemize}
		\item Ausführbare Tätigkeit im Sinne eines Algorithmus
		\item Es muss darauf geachtet werden, welcher Klasse Operationen sinnvoll zugeordnet werden.\\
		Klasse Auto sollte die Operation ``ausstattungHinzufügen'' zugeordnet haben und nicht die Klasse Ausstattung.
		\item Notation Klassenoperation: \uline{operation()}, auch Konstruktoren
		\item Parameter kann Richtung haben (in, out, inOut)
	\end{itemize}
\end{frame}

\begin{frame}
\frametitle{Beispiel Operationen}
	Beispiel
	\begin{itemize}
		\item +halloWelt()
		\item +add(summand1, summand2):int
		\item -div(divident:int, divisor:int):double
		\item \#sub(in minuend:double, in subtrahend:double, out resultat:double)
		\item inc(inOut wert)
		\item setLstKl(lohnsteuerKlasse:int=1)
	\end{itemize}
\end{frame}

\begin{frame}
\frametitle{Assoziationen}
	\begin{block}{Definition}
		Eine Assoziation drückt die Möglichkeit aus Instanzen einer Klasse mit Instanzen derselben oder
		anderer Klassen zu verbinden.
	\end{block}
	\begin{itemize}
		\item Bennenung ist optional, kurze prägnante Verben
		\item Bezeichnung von Rollen der Klassen in einer Assoziation ist optional, Verwendung von Substantiven
	\end{itemize}
\end{frame}

\begin{frame}
\frametitle{Beispiel Assoziation}
	Beispiel
	\center
	\includegraphics[width=0.8\textwidth,
	keepaspectratio=true]{bilder/assoziation.png}
\end{frame}

\begin{frame}
\frametitle{Multiplizitäten Assoziation}
	Beispiel
	\center
	\includegraphics[width=0.8\textwidth,
	keepaspectratio=true]{bilder/asso_multi.png}
	\bigskip
	Eine Kann-Assoziation hat als Untergrenze die Multiplizität 0.\\
	Eine Muss-Assoziation hat die Multiplizität größer 1 oder eine Untergrenze größer 1
\end{frame}

\begin{frame}
\frametitle{Assoziative Klassen}
	Eine Assoziation kann eine Klasse mit Attributen und Operationen darstellen
	\center
	\includegraphics[width=0.8\textwidth,
	keepaspectratio=true]{bilder/asso_klasse.png}
\end{frame}

\begin{frame}
\frametitle{Assoziative Klassen}
	\center
	\includegraphics[width=0.8\textwidth,
	keepaspectratio=true]{bilder/asso_klasse_detail.png}
\end{frame}

\begin{frame}
\frametitle{Reflexive Assoziation / Oder-Beziehung}
	Eine reflexive Assoziation besteht zwischen Objekten derselben Klasse\\
	\center
	\includegraphics[width=0.6\textwidth,
	keepaspectratio=true]{bilder/asso_reflexiv.png}
\end{frame}

\begin{frame}
\frametitle{Navigierbarkeit}
	Die folgende Assoziation definiert eine Beziehung ohne Angaben zu ihrer Navigierbarkeit\\
	\center
	\includegraphics[width=0.6\textwidth,
	keepaspectratio=true]{bilder/asso_navi1.png}
\end{frame}

\begin{frame}
\frametitle{Unidirektionale Navigierbarkeit}
	Instanzen der Klasse Partyteilnehmer kennen Instanzen der Klasse
	Partybegleitung. Die potenzielle Partybegleitung kennt jedoch nicht
	den Partyteilnehmer.
	\\
	Die bidirektionale Navigierbarkeit weist entsprechend Pfeile an beiden
	Assoziationsenden auf.
	\\
	\center
	\includegraphics[width=0.6\textwidth,
	keepaspectratio=true]{bilder/asso_navi2.png}
\end{frame}

\begin{frame}
\frametitle{Aggregation und Komposition}
	\begin{itemize}
		\item Aggregation ist eine Teil-Ganzes-Beziehung, die sich durch ``besteht aus''
		- für das Assoziationsende des Ganzen - und ``ist Teil von'' für das Ende der Teile
		- beschreiben lässt.
	\end{itemize}
	\includegraphics[width=0.6\textwidth,
	keepaspectratio=true]{bilder/asso_agg.png}
	\begin{itemize}
		\item Komposition ist eine starke Aggregation
		\item Die Lebensdauer der Teile ist in diejenige des Ganzen eingebettet
	\end{itemize}
	\includegraphics[width=0.6\textwidth,
	keepaspectratio=true]{bilder/asso_comp.png}
\end{frame}

\begin{frame}
\frametitle{Vererbung}
	Vererbung bedeutet, dass eine spezialisierte Klasse über die Eigenschaften,
	das Verhalten und die Assoziationen einer oder mehrerer allgemeiner Klassen
	verfügen kann.
	\\
	\center
	\includegraphics[width=0.2\textwidth,
	keepaspectratio=true]{bilder/asso_vererbung.png}
\end{frame}

\begin{frame}
\frametitle{Übung 4.6}
	In einer Schule soll der Lehrbetrieb rechnergestützt verwaltet werden:
	\begin{enumerate}
		\item Jeder Lehrer kann bis zu vier Fächer unterrichten
		\item Eine Klasse wird von verschiedenen Lehrern in unterschiedlichen Fächern unterrichtet
		\item Jeder Klasse ist ein bestimmter Lehrer als Klassenlehrer zugeordnet
		\item Jeder Lehrer darf nur für eine Klasse Klassenlehrer sein
		\item Jede Unterrichtsstunde findet in einem bestimmten Raum zu einer bestimmten Zeit statt
		und wird von einem Lehrer vor einer Klasse abgehalten
		\item Jede Klasse hat zwischen 30 und 35 Unterrichtsstunden
		\item Es gibt Fächer die nicht immer unterrichtet werden können
	\end{enumerate}
\end{frame}
\ifloesung
\begin{frame}
\frametitle{Übung 4.6 - Lösung}
	\includegraphics[width=1\textwidth,
	keepaspectratio=true]{bilder/klassend_aufg6_lsg.png}
\end{frame}
\fi
