% \part{Java}

% ----------------------------------------------------------------------------
\section{Projektmanagement}
\begin{frame}[fragile]
	\frametitle{Projektmanagement}
\huge Projektmanagement
\end{frame}

\begin{frame}
\frametitle{Erfolgskriterien}
	Es gibt einige, allgemeine Erfolgskriterien die alle Projekte teilen
	\begin{itemize}
		\item Software wird innerhalb der definierten Zeit fertiggestellt
		\item Kosten liegen innerhalb der Budgetplanung
		\item Gelieferte Software entspricht der Erwartung des Kunden
		\item Entwicklungsteam arbeitet kohärent und effizient zusammen
	\end{itemize}
\end{frame}

\begin{frame}
\frametitle{Charakteristika von Softwareprojekten}
	Softwareprojekte haben bestimmte Charakteristika die sie von Projekten
	in anderen Bereichen unterscheiden
	\begin{itemize}
		\item Software ist intangibel
		\item Große SW-Projekte sind in Technologie, Vorgehen, Organisation
		einzigartig
		\item Prozesse und Vorgehensmodelle sind flexibel und organisationsspezifisch
	\end{itemize}
\end{frame}

\begin{frame}
\frametitle{Managementfaktoren in SW-Projekten}
	\begin{itemize}
		\item Größe der Organisation
		\item Kundentyp
		\item Komplexität der Software
		\item Softwaretyp
		\item Unternehmenskultur
		\item Softwareentwicklungsprozesse
	\end{itemize}
\end{frame}

\begin{frame}
\frametitle{Fundamentale PM Aktivitäten}
	\begin{itemize}
		\item Projektplanung
		\item Risikomanagement
		\item Personalführung
		\item Berichterstattung
		\item Projektierung
	\end{itemize}
\end{frame}

\subsection{Risikomanagement}
\begin{frame}
\frametitle{Risikomangement}
\huge Risikomangement
\end{frame}

\begin{frame}
\frametitle{Risikomanagement}
	\begin{block}{Risiko}
		Ein Problem, das noch nicht eingetreten ist, aber wichtige Projektziele
		oder Projektergebnisse gefährdet, falls es eintritt. Ob es eintreten wird
		kann nicht sicher vorausgesagt werden.
	\end{block}
	\begin{itemize}
		\item In jedem Projekt treten Probleme auf die die Projektziele gefährden
		\item Risikomanagement versucht mögliche Probleme frühzeitig zu identifizieren
		und Maßnahmen einzuleiten
		\item Risikomanagement ist eine kontinuierliche Aktivität und umfasst
		\begin{itemize}
			\item Identifikation von Risiken
			\item Analyse und Bewertung der Risiken
			\item Planung von Gegenmaßnahmen
			\item Risikoüberwachung
		\end{itemize}
	\end{itemize}
\end{frame}

\begin{frame}
\frametitle{Identifikation von Risiken}
	\begin{itemize}
		\item Es kann zwischen Kernrisiken und projektspezifischen Risiken unterschieden werden.
		\item Projektspezifischre Risiken können beispielsweise im Rahmen eines moderierten
		Workshops erhoben werden
	\end{itemize}
\end{frame}

\begin{frame}
\frametitle{Übung 3.1}
	Überlegen Sie, welche Kernrisiken für einen Großteil der Soll-Ist-Abweichungen
	in Softwareprojekten verantwortlich ist.
\end{frame}

\ifloesung
\begin{frame}
\frametitle{Übung 3.1 - Lösung}
	\begin{itemize}
		\item Unklare bzw. sich laufend ändernde Projektziele und Anforderungen
		\item Nicht korrekter Zeitplan
		\item Mitarbeiterfluktuation im Projektteam
		\item Mangelnde Skills der Mitarbeiter
		\item Fehlende Unterstützung im Management
		\item Organisatorische Änderungen
		\item \ldots
	\end{itemize}
\end{frame}

\subsection{Qualitaetssicherung}
\begin{frame}
\frametitle{Qualitaetssicherung}
\huge Qualitaetssicherung
\end{frame}

\subsection{Messen/Bewerten}
\begin{frame}
\frametitle{Messen und Bewerten}
\huge Messen und Bewerten
\end{frame}
