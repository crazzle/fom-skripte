\section{I/O}
\begin{frame}
\frametitle{Ein- und Ausgabe mit Java-IO}
	\huge Ein- und Ausgabe mit Java-IO
\end{frame}

\begin{frame}
\frametitle{Ein- und Ausgabe mit Java-IO}
	Die Ein- und Ausgabe wird in Java mit Streams gelöst.
	\\ \vspace{0,3cm}
	Dabei können als Datenquellen bzw. -Senken 
	\begin{itemize}
	  \item lokale Dateien (Quelle und Senke)
	  \item Dateien im Internet (Quelle und Senke)
	  \item der Bildschirm (Senke)
	  \item oder auch die Tastatur (Quelle)
	\end{itemize}
	verwendet werden.
\end{frame}

\begin{frame}
\frametitle{Ein- und Ausgabe mit Java-IO}
	\begin{block}{Stream}
	Ein Stream ist eine geordnete Folge von Bytes (Bytestrom).
	\begin{itemize}
	  \item Kommt der Stream aus einer Datenquelle heißt er ''InputStream''.
	  \item Mündet er in einer Datensenke wird er ''OutputStream'' genannt.
	  \item Streams sind im Package ''java.io'' zusammengefasst.
	\end{itemize}
	\end{block}
	\center
	\includegraphics[width=0.95\textwidth,
	keepaspectratio=true]{bilder/streams.png}
\end{frame}


\begin{frame}
\frametitle{Streams vs. Reader/Writer}
	\begin{itemize}
	  \item Streams arbeiten grundsätzlich byte-orientiert.
	  \item Reader und Writer erweitern die Funktionalität von Streams und
		ermöglichen eine zeichen-orientierte Verarbeitung
	\end{itemize}
\end{frame}

\subsection{Input}
\begin{frame}[fragile]
	\frametitle{Input}
	\huge Input
\end{frame}

\begin{frame}[fragile]
\frametitle{Dateien als Datenquellen}
	\begin{itemize}
	  \item FileInputStream dient zum Einlesen von Dateien
	  \item Arbeitet mit den Methoden der Klasse ''InputStream''
	  \item read() liest ein byte, dass als int von 0 bis 255 zurückgeliefert wird
	  \item read(byte[] b) liest je nach Arraygröße mehrere Bytes
	  \item Nach Benutzung wird der Stream mit close() geschlossen um alle
	  Ressourcen wieder freizugeben
	\end{itemize}
\end{frame}

\begin{frame}[fragile]
\frametitle{Dateien als Datenquellen}
			\begin{lstlisting}
				public static void main(String[] args) throws IOException {
					// Deklarieren und initialisieren des Streams
					FileInputStream fis = new FileInputStream("resource/TestDatei.txt");
			
					// Unsere Ausgabe
					String ausgabe = "";
			
					// Solange ''r'' nicht -1 ist,
					// ist noch ungelesener Inhalt in der Datei
					byte[] b = new byte[1];
					while (fis.read(b) != -1) {
						// Aktuell gelesenes Byte der Ausgabe anhaengen
						ausgabe += (char) (b[0]);
					}
					System.out.println(ausgabe);
			
					// Stream schliessen
					fis.close();
				}
			\end{lstlisting}
\end{frame}

\begin{frame}[fragile]
\frametitle{Dateien als Datenquellen}
	\begin{itemize}
	  \item Reader arbeiten zeichenorientiert
	  \item zusätzlich zu read() besitzen sie read(char[] b) 
	  \item der BufferedReader enthält soger: readLine()
	  \item readLine() liest die Datei Zeilenweise ein und gibt am Ende der Datei
	  ''null'' zurück
	\end{itemize}
\end{frame}

\begin{frame}[fragile]
\frametitle{Dateien als Datenquellen}
	\begin{itemize}
	  \item Reader werden wie folgt instanziiert:\\
	  \begin{lstlisting}
	  	InputStreamReader isr = new InputStreamReader(new
	  	FileInputStream("test.txt"));
	  \end{lstlisting}
	  \item Und der BufferedReader:
	  \begin{lstlisting}
	  	BufferedReader b = new BufferedReader(new InputStreamReader(new FileInputStream("test.txt")));
	  \end{lstlisting}
	  \item  Alternativ kann der Reader auch mit Hilfe der Klasse FileReader
	  angelegt werden:
	  \begin{lstlisting}
		BufferedReader b = new BufferedReader(new FileReader("test.txt"));
	  \end{lstlisting}
	\end{itemize}
\end{frame}

\begin{frame}[fragile]
\frametitle{Dateien als Datenquellen}
	\begin{lstlisting}
		public static void main(String[] args) throws IOException {
				// Deklarieren und initialisieren des Streams
				FileInputStream fis = new FileInputStream("resource/TestDatei.txt");
				
				// Deklarieren und initialisieren des BufferedReaders
				BufferedReader br = new BufferedReader(new InputStreamReader(fis));
		
				// Unsere Ausgabe
				String ausgabe = "";
		
				// solange ''zeile'' nicht null ist
				String zeile = null;
				while ((zeile = br.readLine()) != null) {
					// Aktuell gelesenes Byte der Ausgabe anhaengen
					ausgabe += zeile;
				}
				System.out.println(ausgabe);
		
				// Stream schliessen
				br.close();
			}
	\end{lstlisting}
\end{frame}

\begin{frame}[fragile]
\frametitle{Java Stanardeingabe}
	Java stellt bestimmte Standard-Eingaben zur Verfügung.
	\begin{itemize}
	  \item Eingabe über statisches Attribut ''in'' der Klasse System
	  \item ''in'' ist Referenz auf Objekt vom Typ ''InputStream''
	  \item Zur Eingabe unter der Konsole
	\end{itemize}
\end{frame}

\begin{frame}
	\frametitle{Aufgaben}
	\begin{enumerate}
	  \item Die Datei ''stars.txt'' enthält 20 Zeilen und in jeder Zeile ist eine
	  bestimmte Anzahl an * sowie Zahlen. Geben Sie die Anzahl der Sterne sowie die
	  Summe der in der Datei enthaltenen Zahlen, je Zeile mit jeweiligen
	  Zeilennummern, aus.
	  \item Bisher verwendeten Sie die Klasse ''Tastatur'' zur Eingabe von
	  Informationen auf der Konsole. Schreiben Sie diese Klasse selbst, mithilfe
	  des Standard-Inputstreams ''System.in''. Ermöglichen Sie dabei das Lesen von
	  Strings, Integer und Floats.
	\end{enumerate}
\end{frame}


\subsection{Output}
\begin{frame}[fragile]
	\frametitle{Output}
	\huge Output
\end{frame}

\begin{frame}[fragile]
\frametitle{Dateien als Datensenken}
	\begin{itemize}
	  \item FileOutputStream dient zum Schreiben in Dateien
	  \item Arbeitet mit den Methoden der Klasse ''OutputStream''
	  \item write(int b) schreibt ein byte
	  \item flush() schreibt gepufferte Daten
	  \item Nach Benutzung wird der Stream mit close() geschlossen um alle
	  Ressourcen wieder freizugeben
	\end{itemize}
\end{frame}

\begin{frame}[fragile]
\frametitle{Dateien als Datenksenken}
		\begin{lstlisting}
				public static void main(String[] args) throws IOException {
					// Deklarieren und initialisieren des Streams
					FileOutputStream fos = new FileOutputStream("resource/TestDatei.txt");
					fos.write('H');
					fos.write('e');
					fos.write('l');
					fos.write('l');
					fos.write('o');
			
					// Puffer schreiben
					fos.flush();
			
					// Schliessen des OutputStreams
					fos.close();
				}
		\end{lstlisting}
\end{frame}

\begin{frame}[fragile]
\frametitle{Dateien als Datenksenken}
	\begin{itemize}
	  \item Writer arbeiten zeichenorientiert
	  \item zusätzlich zu write() besitzen sie write(char[] b) und write(String s)
	  \item der PrintWriter enthält soger: println()
	  \item println() schreibt zeilenweise in die Datei
	\end{itemize}
\end{frame}

\begin{frame}[fragile]
\frametitle{Dateien als Datenksenken}
	\begin{itemize}
	  \item Writer werden wie folgt instanziiert:\\
	  \begin{lstlisting}
	  	OutputStreamWriter osw = new
	  	OutputStreamWriter(new FileOutputStream("resource/TestDatei.txt"));
	  \end{lstlisting}
	  \item Und der PrintWriter:
	  \begin{lstlisting}
	  	PrintWriter pw = new PrintWriter(new OutputStreamWriter(
				new FileOutputStream("resource/TestDatei.txt")));
	  \end{lstlisting}
	  \item  Alternativ kann der Writer auch mit Hilfe der Klasse FileWriter
	  angelegt werden:
	  \begin{lstlisting}
		PrintWriter pw = new PrintWriter( new FileWriter("resource/TestDatei.txt"));
	  \end{lstlisting}
	\end{itemize}
\end{frame}

\begin{frame}[fragile]
\frametitle{Dateien als Datenksenken}
	\begin{lstlisting}
		public static void main(String[] args) throws IOException {
			// Deklarieren und initialisieren des Streams
			FileOutputStream fos = new FileOutputStream("resource/TestDatei.txt");
	
			// Deklarieren und initialisieren des Writers
			PrintWriter pw = new PrintWriter(new OutputStreamWriter(fos));
	
			// Zeile schreiben
			pw.println("Eine Testzeile.");
	
			// Puffer schreiben
			pw.flush();
	
			// schliessen des Writers
			pw.close();
		}
	\end{lstlisting}
\end{frame}

\begin{frame}[fragile]
\frametitle{Java Standardausgabe}
	Java stellt bestimmte Standard-Ausgaben zur Verfügung.
	\begin{itemize}
	  \item Ausgabe über statisches Attribut ''out'' der Klasse System
	  \item ''out'' ist Referenz auf Objekt vom Typ ''PrintStream''
	  \item Zur Ausgabe auf die Konsole
	\end{itemize}
\end{frame}

\begin{frame}
	\frametitle{Aufgaben}
	\begin{enumerate}
	  \item Schreiben Sie die Ergebnisse der Verarbeitung der Datei
	  ''stars.txt'' in die Datei ''starsgezaehlt.txt''. Schreiben sie dabei zu
	  Beginn der Zeile die Zeilennummer, anschließend durch '','' getrennt die
	  Anzahl der Sterne, sowie die Summe der Zahlen. Dahinter folgt der eigentliche
	  Datei-Inhalt je Zeile.
	\end{enumerate}
\end{frame}
