\section{Threads}
\begin{frame}
\frametitle{Parallele Programmierung mit Java-Threads}
	\huge Parallele Programmierung mit Java-Threads
\end{frame}

\begin{frame}
\frametitle{Threads}
	\begin{exampleblock}{Definition eines Threads}
		Ein Thread ist ein Programmstück, dass parallel zu anderen Programmstücken
		ausgeführt wird.
	\end{exampleblock}
	Beispiele für parallel auszuführende Programmstücke sind
	\begin{itemize}
	  \item Benutzerinteraktionen
	  \item komplexe Berechnungen
	  \item \ldots
	\end{itemize}
	Auf Einprozessorsystemen werden Threads mittels Zeitscheiben und
	möglichst häufigen Wechseln realisiert. (Nicht wirklich parallel)
\end{frame}

\begin{frame}[fragile]
	\frametitle{Erste Schritte}
	\huge Threads erzeugen
\end{frame}
\begin{frame}
\frametitle{Threads erzeugen}
	\begin{itemize}
	  \item Threads sind einfache Java-Klassen (java.lang.Thread)
	  \item Können auf zwei Arten erzeugt werden:
	  \begin{enumerate}
	    \item Unterklasse (OOP)
	    \item Runnable implementieren (OOP)
	  \end{enumerate}
	  \item In der Methode run() steht der parallel zu verarbeitende Code
	  \item Gestartet wird der Thread durch start()
	\end{itemize}
\end{frame}

\begin{frame}[fragile]
	\frametitle{Threads erzeugen durch ableiten von ''Thread''}
	\begin{columns}
		\begin{column}{0.5\textwidth}
			\small
			\begin{itemize}
			  \item MeinThread erbt von Thread (OOP)
			  \item run() wird überschrieben (OOP)
			  \item in Main-Methode wird start() aufgerufen, nicht run()
			\end{itemize}
			\normalsize
		\end{column}
		\begin{column}{0.5\textwidth}
			\begin{lstlisting}
				public class MeinThread extends Thread {
					public void run() {
						System.out.println("MeinThread");
					}
				}
				//...
				public static void main(String[] args) throws IOException {
					MeinThread t = new MeinThread();
					t.start();
				}
			\end{lstlisting}
		\end{column}
	\end{columns}
\end{frame}


\begin{frame}[fragile]
	\frametitle{Threads erzeugen durch implementieren von ''Runnable''}
	\begin{columns}
		\begin{column}{0.5\textwidth}
			\small
			\begin{itemize}
			  \item MeinThreadRunner implementiert Runnable(OOP)
			  \item run() wird überschrieben (OOP)
			  \item In Main-Methode wird Instanz von Thread erzeugt (OOP) und das
			  implementierte Runnable übergeben
			  \item Anschließend wird start() auf den Thread aufgerufen, nicht run()
			\end{itemize}
			\normalsize
		\end{column}
		\begin{column}{0.5\textwidth}
			\begin{lstlisting}
				public class MeinThreadRunner implements Runnable {
					public void run() {
						System.out.println("MeinThreadRunner");
					}
				}				
				//...
				public static void main(String[] args) throws IOException {
					MeinThreadRunner runner = new MeinThreadRunner();
					Thread t = new Thread(runner);
					t.start();
				}
			\end{lstlisting}
		\end{column}
	\end{columns}
\end{frame}

\begin{frame}
\frametitle{Threads erzeugen}
	\begin{itemize}
	  \item Threads laufen unabhängig voneinander
	  \item Gefahr bei gemeinsamen Ressourcen
	  \item Main läuft in eigenem Thread
	  \item Main-Thread wird von Laufzeitsystem erzeugt
	  \item Programm terminiert wenn der letzte Thread terminiert
	  \item Thread terminiert wenn die run()-Methode durchlaufen wurde
	\end{itemize}
\end{frame}

\begin{frame}
\frametitle{Die Methode sleep(long millis)}
	\begin{itemize}
	  \item sleep() bewirkt, dass der aufrufende Thread für die übergebene Zeit
	  schlafen gelegt wird
	  \item schlafende Threads verbrauchen keine Rechenleistung
	\end{itemize}
\end{frame}

\begin{frame}[fragile]
	\frametitle{Asynchrone Aufträge mit join()}
	\begin{itemize}
	  \item join()-Methode wird verwendet um auf Thread-Ende zu warten
	  \item Beispiel: Aufgabenverteilung auf mehrere Threads und anschließendes
	  Zusammenfügen der Ergebnisse
	\end{itemize}
\end{frame}

\begin{frame}
	\frametitle{Aufgaben}
	\begin{enumerate}
	  \item Zählen Sie in einem Array vom Typ ''boolean'' alle Felder deren Wert
	  ''true'' ist. Verteilen Sie diese Aufgabe auf eine bestimmte Anzahl von
	  Zähler-Threads.\\
	  \begin{itemize}
	    \item Das Array besitzt 200000000 Felder
	    \item Jeder Thread zählt in einem bestimmten Bereich des Arrays
	    \item Testen Sie die Anwendung ohne, mit 2, 10 und 100 Zähler-Threads
	    \item Wie wirkt sich die Simulation einer komplexen Berechnung in Form von
	    sleep() auf die ansynchrone Beauftragung aus?
	  \end{itemize}
	\end{enumerate}
\end{frame}

\subsection{Synchronisation}
\begin{frame}[fragile]
	\frametitle{Synchronisation und gegenseitiger Ausschluss}
	\huge Synchronisation und gegenseitiger Ausschluss
\end{frame}

\begin{frame}
\frametitle{Synchronisation}
	\begin{itemize}
		\item Das Schlüsselwort ''synchronized'' dient als Sperre für Ressourcen
		\item Sperrt Blöcke indem bestimmtes Objekt gelocked wird (OOP)
		\item Kann Zugriff auf alle Methoden eines Objekts locken \\
		(locken der eigenen Instanz)
		\item synchronized führt Informationen von einem konsistenten Zustand in einen
		anderen konsistenten Zustand
	\end{itemize}
\end{frame}

\begin{frame}[fragile]
\frametitle{Beispiel für synchronize}
	\begin{itemize}
	  \item Gegeben sind zwei Threads: zaehler1 und zaehler2
	  \item Beide Threads erhöhen den Wert eines gemeinsamen Integers
	\end{itemize}
\end{frame}

\begin{frame}[fragile]
\frametitle{Beispiel für synchronize}
\begin{lstlisting}
// Die run()-Methode unserer Runnable-Implementierung
public void run() {
	while (true) {
		MainMitInt.inc();
	}
}
// .....................................................
public class MainMitInt {
	// gemeinsamer Integer
    static int zahl = 0;
    //gemeinsam genutzte Methode
	public static void inc(){
		int neueZahl = zahl + 1;
	    zahl = neueZahl;
	}
	// Main-Methode
    public static void main(String[] args) {
		// Zaehler-Runnable instanziieren
		Thread zaehler1 = new Thread(new Zaehler());
		Thread zaehler2 = new Thread(new Zaehler());

		// Threads starten
		zaehler1.start();
		zaehler2.start();

		// Ausgabe
		while (true) {
		    System.out.println(zahl);
		}
    }
}
\end{lstlisting}
\end{frame}

\begin{frame}[fragile]
\frametitle{Beispiel für synchronize}
	\begin{block}{Problemszenario bei dieser Implementierung:}
	\small
	\begin{enumerate}
	  \item zaehler1 ruft inc() auf, berechnet neueZahl und lädt Ergebnis (z.B.
	  5) in zahl
	  \item zaehler1 wird suspendiert, zaehler2 wird ausgeführt, führt inc() aus
	  und berechnet neueZahl (5 + 1 = 6)
	  \item zaehler2 wird suspendiert, zaehler1 wird fortgesetzt, berechnet in inc()
	  neueZahl (5 + 1 = 6) und schreibt 6 in Variable zahl
	  \item zaehler1 wird suspendiert, zaehler2 wird fortgesetzt und schreibt in
	  pausiertes inc() neueZahl(= 6) in Variable zahl
	\end{enumerate}
	\end{block}
	\begin{alertblock}{Das Problem}
	\small
	Die gemeinsam genutzte Variable zahl ist weiterhin 6 obwohl sie 7
	sein sollte!
	\end{alertblock}
\end{frame}

\begin{frame}[fragile]
	\frametitle{Anwendung von synchronize}
	\begin{columns}
		\begin{column}{0.7\textwidth}
			\small
			\begin{itemize}
			  \item Gemeinsam genutze Variable darf immer nur von einem Thread
			  gleichzeitig benutzt werden
			  \item Sichergestellt durch synchronized im Methodenkopf
			  \item Wird auch als ''Gegenseitiger Ausschluss'' bezeichnet
			  \item Methode kann somit nur von einem Thread gleichzeitig aufgerufen
			  werden
			  \item Andere Threads werden blockiert bis aktueller Thread fertig ist
			  \item Besitzt Objekt weitere synchronized-Methoden, so sind sie auch
			  gelocked
			\end{itemize}
			\normalsize
		\end{column}
		\begin{column}{0.3\textwidth}
			\begin{lstlisting}
				// gemeinsam genutzte Methode
				// versehen mit synchronized
				public static synchronized void inc(){
					int neueZahl = zahl + 1;
				    zahl = neueZahl;
				}
			\end{lstlisting}
		\end{column}
	\end{columns}
\end{frame}

\begin{frame}[fragile]
	\frametitle{Wann synchronize verwenden?}
	\begin{exampleblock}{Wann sollte synchronized verwendet werden:}
	Synchronisation ist immer nur dann notwendig, wenn mehrere Threads auf
	gemeinsame Daten zugreifen und mindestens einer dieser Threads die Daten
	verändert. In diesem Fall ist es wichtig, alle Methoden, die auf die Daten
	zugreifen, als synchronized zu kennzeichnen, gleichgültig, ob die Daten in
	einer Methode nur gelesen oder auch geändert werden.
	\end{exampleblock}
\end{frame}

\begin{frame}
	\frametitle{Aufgaben}
	\begin{enumerate}
	  \item Entwerfen Sie ein Bankkonto das Thread-Safe ist
	  \begin{itemize}
	    \item Das Konto besitzt die Methode buchen(int betrag)
	    \item In der Methode wird der übergebene Betrag auf den aktuellen
	    Kontostand aufaddiert
	    \item Mehrere ''Einzahler'' können die Methode aufrufen und Geld einzahlen
	  \end{itemize}
	\end{enumerate}
\end{frame}

\subsection{Wait/Notify}
\begin{frame}[fragile]
	\frametitle{wait() und notify()}
	\huge wait() und notify()
\end{frame}

\begin{frame}[fragile]
	\frametitle{Nebenbedingungen}
	Bedingung fuer Ausführung einer Methode wird erweitert:
	\begin{itemize}
	  \item zusätzlich zu konsistenten Zustand
	  \item wird auf Erfüllung von Nebenbedingung gewartet
	  \item Thread soll mit Methodenausführung warten, bis diese Bedingungen
	  erfüllt sind
	\end{itemize}
\end{frame}

\begin{frame}[fragile]
	\frametitle{Beispielszenario: Parkhaus}
	\begin{block}{Parkhaus das Anzahl noch freier Parkplätze managed:}
	\begin{itemize}
	  \item Verschiedene Threads können Parkhaus passieren() und verlassen()
	  \item Bei Einfahrt wird Anzahl freier Plätze vermindert \\
	  (0 wenn keiner mehr frei, Parkhaus voll, keine Einfahrt möglich)
	  \item Bei Ausfahrt wird Anzahl freier Platze erhöht
	  \item Da Parkhaus von mehreren Auto-Threads genutzt, und zustand geändert
	  werden kann, müssen einfahren() und passieren() synchronized sein
	\end{itemize}
	\end{block}
	\begin{alertblock}{Problem:}
	Wie wird das Warten auf einen freien Platz gelöst?
	\end{alertblock}
\end{frame}

\begin{frame}[fragile]
	\frametitle{Wait-Notify zur Lösung des Problems}
	Zur Lösung werden die Methoden wait() und notify() verwendet
	\begin{itemize}
	  \item wait() blockiert aufrufenden Thread und packt ihn in Warteschlange des
	  Objekts auf das wait() aufgerufen wird
	  \item wait() gibt locks auf Objekt frei
	  \item notify() entfernt Thread aus der Warteschlange 
	  \item notify() hat auf leere Warteschlange keine Wirkung
	  \item keine Garantie das der Thread der am längsten wartet, geweckt wird
	\end{itemize}
\end{frame}

\begin{frame}
	\frametitle{Aufgaben}
	\begin{enumerate}
	  \item Implementieren Sie die Parkhaus-Simulation
	  \begin{itemize}
	    \item Entwerfen Sie die Klasse Parkhaus (OOP)
	    \item Implementieren Sie die Methoden passieren() und verlassen()
	    \item Implementieren Sie die Auto-Threads die nach und nach das Parkhaus
	    verlassen und passieren
	    \item Die Auto-Threads rufen dazu jeweils passieren() oder verlassen() auf
	    dem Parkaus auf
	  \end{itemize}
	\end{enumerate}
\end{frame}

\subsection{Semaphoren}
\begin{frame}[fragile]
	\frametitle{Semaphoren}
	\huge Semaphoren
\end{frame}

\begin{frame}[fragile]
	\frametitle{Semaphoren}
	\begin{exampleblock}{Definition}
	Eine Semaphore ist ein Verwaltungsobjekt, dass den Zugriff mehrerer
	Threads/Prozesse auf eine gemeinsame Ressource kontrolliert.
	\end{exampleblock}
	\begin{itemize}
	  \item Parkhaus entspricht der Struktur einer Semaphore
	  \item Semaphoren dienen der Verwaltung beschränkter Ressourcen auf mehrere
	  Prozesse
	  \item Bei Verwendung besitzt jeder Thread die Semaphore als Attribut
	  \item Verwendet für gegenseitigen Ausschluss
	\end{itemize}
\end{frame}

\begin{frame}[fragile]
	\frametitle{Semaphore als Quelltext}
	\begin{lstlisting}
	public class Semaphore {
	private int value = 0;
	
	// Konstruktor (OOP)
	public Semaphore(int val){
		if(val > 0){
			value = val;
		}
	}
	
	// passieren() aus dem Parkhaus
	public synchronized void p(){
		while(value == 0){
			try{
				wait();
			}catch(InterruptedException e){
				// left blank
			}
		}
		value--;
	}
	
	// Verlassen aus dem Parkhaus
	public synchronized void v(){
		value++;
		notify();
	}
}
	\end{lstlisting}
\end{frame}

\begin{frame}[fragile]
	\frametitle{Anwendung der Semaphore}
	\begin{itemize}
	  \item mit p() kann Thread kritischen Bereich betreten
	  \item mit v() kann Thread kritischen Bereich verlassen
	  \item value gibt an wie viele Threads maximal gleichzeitg den kritischen
	  Bereich betreten dürfen
	\end{itemize}
\end{frame}

\begin{frame}[fragile]
	\frametitle{Anwendungsbeispiel}
	\begin{lstlisting}
	// Semaphore die anderweitig bereits initialisiert wurde
	Semaphore s;

	// Unsere run-Methode
	public void run() {
		while (true) {
			// Wir betreten einen kritischen Bereich
			s.p();

			this.doCriticalStuff();

			// Wir verlassen den kritischen Bereich
			s.v();
		}
	}
	\end{lstlisting}
\end{frame}

\subsection{Deadlocks}
\begin{frame}[fragile]
	\frametitle{Deadlocks}
	\huge Deadlocks
\end{frame}

\begin{frame}[fragile]
	\frametitle{Deadlocks}
	Ein Deadlock ist eine Situation, in der sich zwei oder mehr Threads in
	einem dauernden Wartezustand befinden.
\end{frame}

\begin{frame}[fragile]
	\frametitle{Situation}
	Zwei Köche, benötigen Schüssel und Löffel zum Kochen
	\begin{itemize}
	  \item Koch 1 nimmt zuerst Löffel und dann Schüssel
	  \item Koch 2 nimmt zuerst Schüssel und dann Löffel
	\end{itemize}
\end{frame}

\begin{frame}[fragile]
	\frametitle{Szenario}
	\begin{block}{Folgendes Szenario entsteht:}
	\begin{enumerate}
	  \item Koch 1 nimmt Löffel, Thread wird unterbrochen
	  \item Koch 2 nimmt Schüssel, Thread wird unterbrochen
	  \item Koch 1 wartet darauf das Schüssel frei wird
	  \item Koch 2 wartet darauf das Löffel frei wird
	\end{enumerate}
	\end{block}
	\begin{alertblock}{Das Problem:}
	Deadlock!
	\end{alertblock}
\end{frame}

\begin{frame}[fragile]
	\frametitle{Die vier Bedingungen für Deadlocks}
	\begin{enumerate}
	  \item Wenn Ressource nur unter Ausschluss nutzbar
	  \item Genutzte Ressourcen können nutzendem Thread nicht entzogen werden
	  \item Threads besitzen Ressourcen und fordern weitere an
	  \item Es existiert zyklische Kette von Threads, von denen jeder mindestens
	  eine Ressource besitzt, die der nächste Thread in der Kette benötigt
	\end{enumerate}
\end{frame}

\begin{frame}[fragile]
	\frametitle{Vermeidung von Deadlocks}
	\begin{itemize}
	  \item Thread darf nur Ressourcen anfordern, wenn er keine Besitzt
	  \item Thread forder Ressourcen in bestimmter Reihenfolge an um zyklische
	  Abhängigkeiten zu verhindern
	  \item Prüfung ob es bei Ressourcenanforderung zu einem Deadlock kommen kann
	\end{itemize}
\end{frame}