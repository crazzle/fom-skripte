\section{OOP}
\begin{frame}
\frametitle{Objektorientierte Programmierung}
\begin{block}{Blocktitel}
Blocktext 
\end{block}

\begin{exampleblock}{Blocktitel}
Blocktext 
\end{exampleblock}


\begin{alertblock}{Blocktitel}
Blocktext 
\end{alertblock}

\end{frame}

% http://www.hki.uni-koeln.de/sites/all/files/courses/5779/instanz_klasse.gif

\begin{itemize}
  \item Source Code eines Java-Programms steht in einer Textdatei mit der
		Endung .java
  \item Datei wird durch Compiler (javac) in .class Datei \"ubersetzt
  \item Class-Datei enth\"alt Bytecode, der in JVM (Interpreter) l\"auft\\
  \item JVM wird durch den Befehl ''java'' aufgerufen und f\"uhrt den
		Bytecode aus
  \item Die JVM ist Teil des Java Runtime Environment (JRE), das es f\"ur jedes
  		g\"angige Betriebssystem gibt
  \item F\"ur verschiedene Anwendungsgebiete existieren verschiedene
  		Plattformen:\\
  		J2SE, J2ME, J2EE
\end{itemize}

\begin{frame}[fragile]
	  \frametitle{Attribute}
	  \begin{block}{Attribute bestehen aus}
		  \begin{enumerate}
		  	\item Datentyp
		  	\item Variablennamen
		  	\item Standardwert (wenn nicht initialisiert)
		  \end{enumerate}
		  Syntax: Modifier Datentyp Attributname;
	  \end{block}
	  \begin{block}{Es gibt 2 Arten von Datentypen}
		  	\begin{enumerate}
	  			\item elementare Datentypen
	  			\item Referenzen (Verweise auf Objekte)
	  		\end{enumerate}
	  \end{block}
\end{frame}